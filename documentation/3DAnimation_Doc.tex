\documentclass{sig-alternate-05-2015}


\begin{document}


\title{Avalanche Simulation in a Particle System}
\subtitle{As a part of the Master - Module 3D-Animation in the Hochschule Rhein Main
purely written in Python and OpenGL}
%
% You need the command \numberofauthors to handle the 'placement
% and alignment' of the authors beneath the title.
%
% For aesthetic reasons, we recommend 'three authors at a time'
% i.e. three 'name/affiliation blocks' be placed beneath the title.
%
% NOTE: You are NOT restricted in how many 'rows' of
% "name/affiliations" may appear. We just ask that you restrict
% the number of 'columns' to three.
%
% Because of the available 'opening page real-estate'
% we ask you to refrain from putting more than six authors
% (two rows with three columns) beneath the article title.
% More than six makes the first-page appear very cluttered indeed.
%
% Use the \alignauthor commands to handle the names
% and affiliations for an 'aesthetic maximum' of six authors.
% Add names, affiliations, addresses for
% the seventh etc. author(s) as the argument for the
% \additionalauthors command.
% These 'additional authors' will be output/set for you
% without further effort on your part as the last section in
% the body of your article BEFORE References or any Appendices.

\numberofauthors{2}
\author{
%
% The command \alignauthor (no curly braces needed) should
% precede each author name, affiliation/snail-mail address and
% e-mail address. Additionally, tag each line of
% affiliation/address with \affaddr, and tag the
% e-mail address with \email.
%
% 1st. author
\alignauthor
Tiras Zemicael\\
       \affaddr{Hochschule RheinMain}\\
       \affaddr{1932 Wallamaloo Lane}\\
       \affaddr{Wallamaloo, New Zealand}\\
       \email{Tiras.Zemicael@hotmail.de}
% 2nd. author
\alignauthor
Simon Rininsland\\
       \affaddr{Roseggerstrasse 5, 65187 Wiesbaden}\\
       \email{Simon.Rininsland@student.hs-rm.de}
}


\maketitle
\begin{abstract}
Avalanche
A natural dreaded force of many snow and ice particles rushing down a Slope, driven by the Gravity. As many as snowflakes and ice particles which are included in an avalanche as good as we can play with them in an Particle System. One of the best examples for dynamicly rendered simulations for Particle Systems a snow Avalanche will be the central Part in our Project.\\
In order also to start just from the basics we decided to not use huge frameworks and start from the OpenGL Scatch. We will just use OpenGL Basics.\\ 
\\
We will solve some Physically based Problems which comes around with the Topic of an Avalanche like:\\
- Particles with seperated masses, driven by a force.\\
- Physically Effects, bouncing Particles and combining ones.\\
and some OpenGL based Problems like: \\
- shadow for every seperated Particle \\
- performance Issues and optimization.\\ 
\end{abstract}


\begin{CCSXML}
<ccs2012>
<concept>
<concept_id>10010147.10010371.10010352.10010379</concept_id>
<concept_desc>Computing methodologies~Physical simulation</concept_desc>
<concept_significance>300</concept_significance>
</concept>
</ccs2012>
\end{CCSXML}

\ccsdesc[300]{Computing methodologies~Physical simulation}



\keywords{ACM proceedings; \LaTeX; text tagging}

\section{Introduction}
The Timeline of Avalanche Simulation is as big as the benefit we get from this simulations. With the help of this Simulation protection ramparts can be build and avlanche breakers can put in the optimal position. Not only usefull aspects of Avalanche Simulation should be mentioned, also the esthetic Aspect is a huge one in 3D-Animation. 

\bibliographystyle{abbrv}
\bibliography{sigproc} 

\end{document}
